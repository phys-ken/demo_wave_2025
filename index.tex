% Options for packages loaded elsewhere
\PassOptionsToPackage{unicode}{hyperref}
\PassOptionsToPackage{hyphens}{url}
\PassOptionsToPackage{dvipsnames,svgnames,x11names}{xcolor}
%
\documentclass[
  letterpaper,
]{bxjsbook}

\usepackage{amsmath,amssymb}
\usepackage{iftex}
\ifPDFTeX
  \usepackage[T1]{fontenc}
  \usepackage[utf8]{inputenc}
  \usepackage{textcomp} % provide euro and other symbols
\else % if luatex or xetex
  \usepackage{unicode-math}
  \defaultfontfeatures{Scale=MatchLowercase}
  \defaultfontfeatures[\rmfamily]{Ligatures=TeX,Scale=1}
\fi
\usepackage{lmodern}
\ifPDFTeX\else  
    % xetex/luatex font selection
\fi
% Use upquote if available, for straight quotes in verbatim environments
\IfFileExists{upquote.sty}{\usepackage{upquote}}{}
\IfFileExists{microtype.sty}{% use microtype if available
  \usepackage[]{microtype}
  \UseMicrotypeSet[protrusion]{basicmath} % disable protrusion for tt fonts
}{}
\makeatletter
\@ifundefined{KOMAClassName}{% if non-KOMA class
  \IfFileExists{parskip.sty}{%
    \usepackage{parskip}
  }{% else
    \setlength{\parindent}{0pt}
    \setlength{\parskip}{6pt plus 2pt minus 1pt}}
}{% if KOMA class
  \KOMAoptions{parskip=half}}
\makeatother
\usepackage{xcolor}
\setlength{\emergencystretch}{3em} % prevent overfull lines
\setcounter{secnumdepth}{5}
% Make \paragraph and \subparagraph free-standing
\makeatletter
\ifx\paragraph\undefined\else
  \let\oldparagraph\paragraph
  \renewcommand{\paragraph}{
    \@ifstar
      \xxxParagraphStar
      \xxxParagraphNoStar
  }
  \newcommand{\xxxParagraphStar}[1]{\oldparagraph*{#1}\mbox{}}
  \newcommand{\xxxParagraphNoStar}[1]{\oldparagraph{#1}\mbox{}}
\fi
\ifx\subparagraph\undefined\else
  \let\oldsubparagraph\subparagraph
  \renewcommand{\subparagraph}{
    \@ifstar
      \xxxSubParagraphStar
      \xxxSubParagraphNoStar
  }
  \newcommand{\xxxSubParagraphStar}[1]{\oldsubparagraph*{#1}\mbox{}}
  \newcommand{\xxxSubParagraphNoStar}[1]{\oldsubparagraph{#1}\mbox{}}
\fi
\makeatother


\providecommand{\tightlist}{%
  \setlength{\itemsep}{0pt}\setlength{\parskip}{0pt}}\usepackage{longtable,booktabs,array}
\usepackage{calc} % for calculating minipage widths
% Correct order of tables after \paragraph or \subparagraph
\usepackage{etoolbox}
\makeatletter
\patchcmd\longtable{\par}{\if@noskipsec\mbox{}\fi\par}{}{}
\makeatother
% Allow footnotes in longtable head/foot
\IfFileExists{footnotehyper.sty}{\usepackage{footnotehyper}}{\usepackage{footnote}}
\makesavenoteenv{longtable}
\usepackage{graphicx}
\makeatletter
\newsavebox\pandoc@box
\newcommand*\pandocbounded[1]{% scales image to fit in text height/width
  \sbox\pandoc@box{#1}%
  \Gscale@div\@tempa{\textheight}{\dimexpr\ht\pandoc@box+\dp\pandoc@box\relax}%
  \Gscale@div\@tempb{\linewidth}{\wd\pandoc@box}%
  \ifdim\@tempb\p@<\@tempa\p@\let\@tempa\@tempb\fi% select the smaller of both
  \ifdim\@tempa\p@<\p@\scalebox{\@tempa}{\usebox\pandoc@box}%
  \else\usebox{\pandoc@box}%
  \fi%
}
% Set default figure placement to htbp
\def\fps@figure{htbp}
\makeatother
% definitions for citeproc citations
\NewDocumentCommand\citeproctext{}{}
\NewDocumentCommand\citeproc{mm}{%
  \begingroup\def\citeproctext{#2}\cite{#1}\endgroup}
\makeatletter
 % allow citations to break across lines
 \let\@cite@ofmt\@firstofone
 % avoid brackets around text for \cite:
 \def\@biblabel#1{}
 \def\@cite#1#2{{#1\if@tempswa , #2\fi}}
\makeatother
\newlength{\cslhangindent}
\setlength{\cslhangindent}{1.5em}
\newlength{\csllabelwidth}
\setlength{\csllabelwidth}{3em}
\newenvironment{CSLReferences}[2] % #1 hanging-indent, #2 entry-spacing
 {\begin{list}{}{%
  \setlength{\itemindent}{0pt}
  \setlength{\leftmargin}{0pt}
  \setlength{\parsep}{0pt}
  % turn on hanging indent if param 1 is 1
  \ifodd #1
   \setlength{\leftmargin}{\cslhangindent}
   \setlength{\itemindent}{-1\cslhangindent}
  \fi
  % set entry spacing
  \setlength{\itemsep}{#2\baselineskip}}}
 {\end{list}}
\usepackage{calc}
\newcommand{\CSLBlock}[1]{\hfill\break\parbox[t]{\linewidth}{\strut\ignorespaces#1\strut}}
\newcommand{\CSLLeftMargin}[1]{\parbox[t]{\csllabelwidth}{\strut#1\strut}}
\newcommand{\CSLRightInline}[1]{\parbox[t]{\linewidth - \csllabelwidth}{\strut#1\strut}}
\newcommand{\CSLIndent}[1]{\hspace{\cslhangindent}#1}

\usepackage{luatexja}
\usepackage{luatexja-fontspec}
\setmainjfont{IPAexMincho}  % 日本語フォント指定
\setsansjfont{IPAexGothic}
\setmonojfont{IPAGothic}
\makeatletter
\@ifpackageloaded{tcolorbox}{}{\usepackage[skins,breakable]{tcolorbox}}
\@ifpackageloaded{fontawesome5}{}{\usepackage{fontawesome5}}
\definecolor{quarto-callout-color}{HTML}{909090}
\definecolor{quarto-callout-note-color}{HTML}{0758E5}
\definecolor{quarto-callout-important-color}{HTML}{CC1914}
\definecolor{quarto-callout-warning-color}{HTML}{EB9113}
\definecolor{quarto-callout-tip-color}{HTML}{00A047}
\definecolor{quarto-callout-caution-color}{HTML}{FC5300}
\definecolor{quarto-callout-color-frame}{HTML}{acacac}
\definecolor{quarto-callout-note-color-frame}{HTML}{4582ec}
\definecolor{quarto-callout-important-color-frame}{HTML}{d9534f}
\definecolor{quarto-callout-warning-color-frame}{HTML}{f0ad4e}
\definecolor{quarto-callout-tip-color-frame}{HTML}{02b875}
\definecolor{quarto-callout-caution-color-frame}{HTML}{fd7e14}
\makeatother
\makeatletter
\@ifpackageloaded{bookmark}{}{\usepackage{bookmark}}
\makeatother
\makeatletter
\@ifpackageloaded{caption}{}{\usepackage{caption}}
\AtBeginDocument{%
\ifdefined\contentsname
  \renewcommand*\contentsname{Table of contents}
\else
  \newcommand\contentsname{Table of contents}
\fi
\ifdefined\listfigurename
  \renewcommand*\listfigurename{List of Figures}
\else
  \newcommand\listfigurename{List of Figures}
\fi
\ifdefined\listtablename
  \renewcommand*\listtablename{List of Tables}
\else
  \newcommand\listtablename{List of Tables}
\fi
\ifdefined\figurename
  \renewcommand*\figurename{Figure}
\else
  \newcommand\figurename{Figure}
\fi
\ifdefined\tablename
  \renewcommand*\tablename{Table}
\else
  \newcommand\tablename{Table}
\fi
}
\@ifpackageloaded{float}{}{\usepackage{float}}
\floatstyle{ruled}
\@ifundefined{c@chapter}{\newfloat{codelisting}{h}{lop}}{\newfloat{codelisting}{h}{lop}[chapter]}
\floatname{codelisting}{Listing}
\newcommand*\listoflistings{\listof{codelisting}{List of Listings}}
\makeatother
\makeatletter
\makeatother
\makeatletter
\@ifpackageloaded{caption}{}{\usepackage{caption}}
\@ifpackageloaded{subcaption}{}{\usepackage{subcaption}}
\makeatother

\usepackage{bookmark}

\IfFileExists{xurl.sty}{\usepackage{xurl}}{} % add URL line breaks if available
\urlstyle{same} % disable monospaced font for URLs
\hypersetup{
  pdftitle={【物理基礎】波とエネルギー},
  pdfauthor={phys-ken},
  colorlinks=true,
  linkcolor={blue},
  filecolor={Maroon},
  citecolor={Blue},
  urlcolor={Blue},
  pdfcreator={LaTeX via pandoc}}


\title{【物理基礎】波とエネルギー}
\author{phys-ken}
\date{2025-03-02}

\begin{document}
\maketitle

\renewcommand*\contentsname{Table of contents}
{
\hypersetup{linkcolor=}
\setcounter{tocdepth}{2}
\tableofcontents
}

\bookmarksetup{startatroot}

\chapter*{はじめに}\label{ux306fux3058ux3081ux306b}
\addcontentsline{toc}{chapter}{はじめに}

\markboth{はじめに}{はじめに}

物理基礎の波の分野の授業補助資料です。

\href{https://phys-ken.github.io/demo_wave_2025/\%E3\%80\%90\%E7\%89\%A9\%E7\%90\%86\%E5\%9F\%BA\%E7\%A4\%8E\%E3\%80\%91\%E6\%B3\%A2\%E3\%81\%A8\%E3\%82\%A8\%E3\%83\%8D\%E3\%83\%AB\%E3\%82\%AE\%E3\%83\%BC.pdf}{PDFはこちらから}
WEBページの内容を自動でPDF化したものです。
\textbf{授業プリントではありません} 。

\section*{在校生へ}\label{ux5728ux6821ux751fux3078}
\addcontentsline{toc}{section}{在校生へ}

\markright{在校生へ}

\begin{itemize}
\tightlist
\item
  このページは、あくまで授業の補助資料です。演習問題等は著作権の関係で一般公開でウェブ公開することはできません。Google
  classroomの添付資料を確認してください。
\end{itemize}

\section*{このページでよく登場する映像教材}\label{ux3053ux306eux30daux30fcux30b8ux3067ux3088ux304fux767bux5834ux3059ux308bux6620ux50cfux6559ux6750}
\addcontentsline{toc}{section}{このページでよく登場する映像教材}

\markright{このページでよく登場する映像教材}

\begin{enumerate}
\def\labelenumi{\arabic{enumi}.}
\tightlist
\item
  細川JP
\item
  もりぽん
\item
  Try it 物理基礎
\item
  たのしい物理
\item
  マイコンで物理
\end{enumerate}

\bookmarksetup{startatroot}

\chapter{波の表し方}\label{ux6ce2ux306eux8868ux3057ux65b9}

\begin{tcolorbox}[enhanced jigsaw, breakable, rightrule=.15mm, opacitybacktitle=0.6, title=\textcolor{quarto-callout-tip-color}{\faLightbulb}\hspace{0.5em}{このセクションの目標}, coltitle=black, toprule=.15mm, bottomtitle=1mm, bottomrule=.15mm, toptitle=1mm, titlerule=0mm, arc=.35mm, opacityback=0, left=2mm, colback=white, colbacktitle=quarto-callout-tip-color!10!white, leftrule=.75mm, colframe=quarto-callout-tip-color-frame]

\begin{itemize}
\tightlist
\item
  媒質の運動と波の形を関連付けて説明できる。
\item
  波の運動を2種類のグラフで表現できる。
\end{itemize}

\end{tcolorbox}

\section{波の表し方}\label{ux6ce2ux306eux8868ux3057ux65b9-1}

\begin{itemize}
\item
  「振動が伝わっていく現象」を、波と呼びます。以下の映像教材を参考に、波を表す様々な言葉を正しく理解しましょう。
\end{itemize}

\section{波を表すグラフ}\label{ux6ce2ux3092ux8868ux3059ux30b0ux30e9ux30d5}

\begin{itemize}
\item
  波を表すグラフは2通りあります。\textbf{目に見えた波の形をそのままグラフにしたy-xグラフ}と、\textbf{ある一点の位置の変化に注目したy-tグラフ}の2種類です。
\item
  物理の理解のためには、同じ理論や現象を複数の方法で説明することが効果的です。例えば、2種類のグラフについて、このような方法で説明されることもあります。
\end{itemize}

\section{演習問題の取り組み方}\label{ux6f14ux7fd2ux554fux984cux306eux53d6ux308aux7d44ux307fux65b9}

\begin{itemize}
\item
  演習問題はGoogle classroomにアップされています。
\item
  答えを見てもかまいません。しかし、最終目標は演習問題を理解し、一人で回答できるようになることです。
\item
  他人に説明したり、自分の言葉でまとめなおすことはとても重要です。わからないときは、気軽に質問しましょう。その際、どこがわからないかをしっかり説明しましょう。

  \begin{itemize}
  \tightlist
  \item
    「全部わかりません」「初めから教えて」のようなあいまいな質問は、教える側にとっても教わる側にとっても時間の無駄です!「(1)でなんでこの値が0になるのかわからない」「この問題、私は~~だと思うんだけど、どこが間違っているの?」と聞いてください。
  \end{itemize}
\end{itemize}

\bookmarksetup{startatroot}

\chapter{連続波の表し方}\label{ux9023ux7d9aux6ce2ux306eux8868ux3057ux65b9}

\begin{tcolorbox}[enhanced jigsaw, breakable, rightrule=.15mm, opacitybacktitle=0.6, title=\textcolor{quarto-callout-tip-color}{\faLightbulb}\hspace{0.5em}{このセクションの目標}, coltitle=black, toprule=.15mm, bottomtitle=1mm, bottomrule=.15mm, toptitle=1mm, titlerule=0mm, arc=.35mm, opacityback=0, left=2mm, colback=white, colbacktitle=quarto-callout-tip-color!10!white, leftrule=.75mm, colframe=quarto-callout-tip-color-frame]

\begin{itemize}
\tightlist
\item
  連続波を表す量を正しく理解する。
\item
  連続波について、2種類のグラフで正しく表現できる。
\item
  波の関係式\(v=f \lambda\)を正しく適用して、波の速さを求めることができる。
\end{itemize}

\end{tcolorbox}

\section{連続波とは}\label{ux9023ux7d9aux6ce2ux3068ux306f}

\begin{itemize}
\item
  1個、2個\ldots と数えることができるような孤立した波のことをパルス波と呼びます。
  ::: \{.only-format html\}
  \href{http://www.info-niigata.or.jp/~ymiyata/butsuri22/33nami1/nami1_02.html}{\pandocbounded{\includegraphics[keepaspectratio]{images/nami1_fig03.gif}}}
  :::
\item
  それに対して、1個、2個と数えることができないような連続した波を、連続波といいます。
\item
  高校の物理基礎では、正弦波と呼ばれる最もシンプルな連続波を扱います。
\end{itemize}

::: \{.only-format html\}
\href{http://www.info-niigata.or.jp/~ymiyata/butsuri22/33nami1/nami1_02.html}{\pandocbounded{\includegraphics[keepaspectratio]{images/nami1_fig07.gif}}}
:::

\begin{tcolorbox}[enhanced jigsaw, breakable, rightrule=.15mm, opacitybacktitle=0.6, title=\textcolor{quarto-callout-warning-color}{\faExclamationTriangle}\hspace{0.5em}{注意}, coltitle=black, toprule=.15mm, bottomtitle=1mm, bottomrule=.15mm, toptitle=1mm, titlerule=0mm, arc=.35mm, opacityback=0, left=2mm, colback=white, colbacktitle=quarto-callout-warning-color!10!white, leftrule=.75mm, colframe=quarto-callout-warning-color-frame]

「正弦波」は、奥が深い現象です。インターネット等で検索すると、「単振動」と関連付けて説明されたり、「\(y = A \sin \left( 2\pi \left( \frac{x}{\lambda} - \frac{t}{T} \right) \right)\)」のように数式で説明されていることもあります。

物理基礎の学習の段階では細かい言葉遣いはあまり気にせず、「ウネウネした波っぽいやつ」程度に考えてくれれば結構です。

\end{tcolorbox}

\section{正弦波を表す量}\label{ux6b63ux5f26ux6ce2ux3092ux8868ux3059ux91cf}

\begin{itemize}
\tightlist
\item
  下記の映像教材を確認して、波長や周期、振動数といった言葉を理解しましょう。
\end{itemize}

動画内で登場する\(v=f\lambda\)や\(f=\frac{1}{T}\)は、波の分野の学習でよく使う式です。

\section{演習問題}\label{ux6f14ux7fd2ux554fux984c}

\begin{itemize}
\tightlist
\item
  前回学習した波の2種類のグラフに関する知識を使って、正弦波をグラフで表してみましょう。
\item
  演習問題は、Google classroomで配信します。
\end{itemize}

\subsection{参考資料}\label{ux53c2ux8003ux8cc7ux6599}

\begin{itemize}
\tightlist
\item
  ここまでの学習内容を1本にまとめた動画です。後半の「位相」の部分は、物理基礎の単元では扱いません。
\end{itemize}

\bookmarksetup{startatroot}

\chapter{References}\label{references}

Einstein's theory of relativity is groundbreaking (Einstein 1905).

\phantomsection\label{refs}
\begin{CSLReferences}{1}{0}
\bibitem[\citeproctext]{ref-Einstein1905}
Einstein, Albert. 1905. {``On the Electrodynamics of Moving Bodies.''}
\emph{Annalen Der Physik} 17: 891--921.

\end{CSLReferences}




\end{document}
